\chapter{Resource Classes}
\label{resource_classes}

Using \jaxrs\, a Web resource is implemented as a resource class; this section describes resource classes in detail.

\section{URI Templates}

A resource class is anchored in URI space using the \UriTemplate\ annotation. The value of the annotation is a relative URI template with the base URI being provided by the deployment context. Root resource classes are anchored directly using a \UriTemplate\ annotation on the class.

A URI template is a string with zero or more embedded parameters that, when values are substituted for all the parameters, conforms to the URI\cite{uri} production. A parameter is represented as \lq\{\rq{\em name}\lq\}\rq\  where {\em name} is the name of the parameter. E.g.:

\begin{ednote}Add reference to URI Templates ID when available.\end{ednote}

\begin{listing}{1}
@UriTemplate("widgets/{id}")
public class Widget {
  ...
}\end{listing}

In the above example the \code{Widget} resource class is identified by the relative URI \code{widgets/{\em xxx}} where \code{\em xxx} is the value of the \code{id} parameter.

The \code{encode} property controls whether the value of the \UriTemplate\ annotation is automatically encoded (the default) or not. When automatic encoding is disabled, care must be taken to ensure that the value of the URI template is valid. E.g. the following two lines are equivalent:

\begin{listing}{1}
@UriTemplate("widget list/{id}")
@UriTemplate(value="widget%20list/{id}" encode=false)
\end{listing}

The \code{limited} property controls whether a trailing template variable matches a single path segment or multiple. Setting the property to false allows a single template variable to match a path and can be used, e.g., when a template represents a path prefix followed by an arbitrary length path.

\subsection{Sub Resources}
\label{sub_resources}

Resource class methods can also be annotated with \UriTemplate. The effect of the annotation depends on whether the method is also annotated with \HttpMethod\ or not:

\begin{description}
\item[Not annotated with \HttpMethod] Such methods, known as sub-resource locators, are used to further resolve the object that will handle the request. Any returned object is treated as a resource class and used to either handle the request or to further resolve the object that will handle the request, see \ref{mapping_requests_to_java_methods} for further details.
\item[Annotated with \HttpMethod] Such methods, known as sub-resource methods, are treated like a normal resource method (see section \ref{resource_method}) except the method is only invoked for request URIs that match a URI template created by concatenating the URI template of the resource class with the URI template of the method\footnote{If the resource class URI template does not end with a \lq/\rq\ character then one is added during the concatenation.}.
\end{description}

The following example illustrates the difference:

\begin{listing}{1}
@UriTemplate("widgets")
public class WidgetList {
  @HttpMethod
  @UriTemplate("offers")
  WidgetList getDiscounted() {...}
  
  @UriTemplate("{id}")
  Widget findWidget(@UriParam("id") String id) {
    return lookupWidget(id);
  }
}\end{listing}

In the above a \code{GET} request for the \code{widgets/offers} resource is handled directly by the \code{getDiscounted} sub-resource method of \code{WidgetList} whereas a \code{GET} request for \code{widgets/{\em xxx}} is handled by whatever object is returned by the \code{findWidget} sub-resource locator.

Note that a set of sub-resource methods annotated with the same URI template value are functionally equivalent to a similarly annotated sub-resource locator that returns an instance of a resource class with the same set of resource methods.

\section{Lifecycle}

A new resource class instance is created for each request to that resource. First the constructor (see section \ref{resource_class_constructor}) is called, then the appropriate method (see section \ref{resource_method}) is invoked and finally the object is made available for garbage collection.

\section{Constructors}
\label{resource_class_constructor}

Root resource classes are instantiated by the \jaxrs\ runtime and MUST have a constructor with one of the following annotations on every parameter: \HttpContext, \HeaderParam, \MatrixParam, \QueryParam\ or \UriParam. Note that a zero argument constructor is permissible under this rule. Section \ref{resource_method_params} defines the parameter types permitted for each annotation. If more than one constructor that matches the above pattern is available then an implementation MUST use the one with the most parameters. Choosing amongst constructors with the same number of parameters is implementation specific.

Non-root resource classes are instantiated by an application and do not require the above-described constructor.

\section{Resource Methods}
\label{resource_method}

Resource methods are resource class methods annotated with \HttpMethod. They are used to handle requests and MUST conform to certain restrictions described in this section.

The \HttpMethod\ annotation has an optional value that corresponds to the name of a request method. In the absence of a value, the request method is inferred from the resource method name: they match if the resource method name starts with the request method name converted to lower case.

\subsection{Parameters}
\label{resource_method_params}

When the method is invoked, annotated parameter values are mapped from the request according to the semantics of the annotation. The following describes the permitted types for an annotated parameter.
\begin{description}
\item[\MatrixParam, \QueryParam\ or \UriParam] The class of the annotated parameter MUST have a constructor that accepts a single \code{String} argument, or a static method named \code{valueOf} that accepts a single \code{String} argument. By default, parameter values are automatically decoded; automatic decoding can be disabled using the \Encoded\ annotation.
\item[\HttpContext] The class of the annotated parameter MUST be \UriInfo, \PreconditionEvaluator\ or \HttpHeaders. See chapter \ref{context} for additional information on these types.
\item[\HeaderParam] The class of the annotated parameter MUST have a constructor that accepts a single \code{String} argument, or a static method named \code{valueOf} that accepts a single \code{String} argument. Other types may be supported using a \HeaderProvider\ as described in section \ref{header_providers}.
\end{description}

The value of an non-annotated  parameter is mapped from the request entity body. Resource methods MUST NOT have more than one parameter that is not annotated with one of \HttpContext, \HeaderParam, \MatrixParam, \QueryParam\ or \UriParam. Conversion between an entity body and a Java type is the responsibility of an \EntityProvider, see section \ref{entity_providers}.

\subsection{Return Type}
\label{resource_method_return}

Resource methods MAY return \code{void}, \Response\ or another Java type, these return types are mapped to a response entity body as follows:

\begin{description}
\item[\code{void}] Results in an empty entity body.
\item[\code{instanceof} \Response] Results in an entity body mapped from the \code{Entity} property of the \Response.
\item[Other] Results in an entity body mapped from the return type.
\end{description}

Conversion between a Java types and an entity body is the responsibility of an \EntityProvider, see section \ref{entity_providers}.

Methods that need to provide additional metadata with a response should return an instance of \Response, the \Response\code{.Builder} class provides a convenient way to create a \Response\ instance using a builder pattern.

\subsection{Exceptions}

An implementation MUST catch \WebAppExc\ and map it to a response. If the \code{response} property of the exception is not \code{null} then it MUST be used to create the response. If the \code{response} property of the exception is \code{null} an implementation MUST generate a server error response.

An implementation MUST allow other runtime exceptions to propagate to the underlying container. This allows existing container facilities (e.g. a Servlet filter) to be used to handle the error if desired.

\begin{ednote}What to do about checked exceptions ? If we allow them on resource methods then do we need some standard runtime exception that can be used to wrap the checked exception so it can be propagated to the container in a standard way ?\end{ednote}

\subsection{HEAD and OPTIONS}
\label{head_and_options}

\code{HEAD} and \code{OPTIONS} requests receive additional support. On receipt of \code{HEAD} request an implementation MUST either:

\begin{enumerate}
\item Call a method annotated with \HttpMethod\ that supports \code{HEAD} or, if none present,
\item\label{get_not_head} Call a method annotated with \HttpMethod\ that supports \code{GET} and discard any returned entity.
\end{enumerate}

Note that option \ref{get_not_head} may result in reduced performance where entity creation is significant.

On receipt of an \code{OPTIONS} request an implementation MUST either:

\begin{enumerate}
\item Call a method annotated with \HttpMethod\ that supports \code{OPTIONS} or, if none present,
\item Generate an automatic response from the declared metadata of the matching class.
\end{enumerate}


\section{Declaring Media Type Capabilities}
\label{declaring_method_capabilities}

Application classes can declare the supported request and response media types using the \ProduceMime\ and \ConsumeMime\ annotations. These annotations MAY be applied to a resource class method, a resource class, or to an \EntityProvider\ (see section \ref{declaring_provider_capabilities}). Declarations on a resource class method override any on the resource class; declarations on an \EntityProvider\ for a method argument or return type override those on a resource class or resource method. In the absence of either of these annotations, support for any media type (\lq\lq*/*\rq\rq) is assumed.

The following example illustrates the \ProduceMime\ annotation:

\begin{listing}{1}
@UriTemplate("widgets")
@ProduceMime("application/widgets+xml")
public class WidgetList {
  
  @HttpMethod
  String getAll() {...}
  
  @HttpMethod
  @UriTemplate("{id}")
  Widget getWidget(@UriParam("id") String id) {...}

  @HttpMethod
  @UriTemplate("{id}/description")
  @ProduceMime("text/html")
  String getDescription(@UriParam("id") String id) {...}
}

@Provider
@ProduceMime({"application/widgets+xml", "application/json"})
public class WidgetProvider implements EntityProvider<Widget> {...}
\end{listing}

In the above, the \code{getAll} resource method returns a \code{String} in the \code{application/widgets+xml} format, the \code{getDescription} sub-resource method returns a \code{String} as \code{text/html} and the \code{getWidget} sub-resource method return a \code{Widget} instance that can be mapped to either \code{application/widgets+xml} or \code{application/json} using the \code{WidgetProvider} class (see section \ref{entity_providers} for more information on \EntityProvider).

An implementation MUST NOT invoke a method whose effective value of \ProduceMime\ does not match the request \code{Accept} header. An implementation MUST NOT invoke a method whose effective value of \ConsumeMime\ does not match the request \code{Content-Type} header.

\section{Matching Requests to Resource Methods}
\label{mapping_requests_to_java_methods}

Matching of requests to resource methods proceeds in two stages:

\begin{enumerate}
\item Obtain the object that will handle the request.

\begin{enumerate}
\item Set \code{uri} to the request URI

\item For each resource class compute a regular expression from its URI template using the process described in section \ref{template_to_regex}. If the resource class has sub-resources (see section \ref{sub_resources}) then append \lq(/.*)?\rq\ to the resulting regular expression, if not then append \lq(/)?\rq.

\item Filter the set of resource classes by rejecting those whose regular expression does not match \code{uri}. If the set is empty then no matching resource can be found, the algorithm terminates and an implementation MUST generate a not found response (HTTP 404 status).

\item Sort the set of matching resource classes using the number of characters in the regular expression not resulting from template variables as the primary key and the number of matching groups as a secondary key.

\item Select the first matching class, instantiate an object of that class and set \code{uri} to the value of the final matching group.

\item\label{check_uri} If \code{uri} is empty or is \lq/\rq\ go to step \ref{find_method}.

\item For each of the object's sub-resource methods (see section \ref{sub_resources}) compute a regular expression for the URI template using the process described in section \ref{template_to_regex} and then appending \lq(/)?\rq. If \code{uri} matches any of the regular expressions go to step \ref{find_method}.

\item For each of the object's sub-resource locators (see section \ref{sub_resources}) compute a regular expression for the URI template using the process described in section \ref{template_to_regex} and then appending \lq(/.*)?\rq.

\item Filter the set of sub-resource locators by rejecting those whose regular expression does not match \code{uri}. If the set is empty then no matching resource can be found, the algorithm terminates and an implementation MUST generate a not found response (HTTP 404 status).

\item Sort the set of matching sub-resource locators using the number of matching groups as the primary key and the number of characters in the regular expression as a secondary key.

\item Set \code{uri} to the value of the final matching group and invoke the first matching method to obtain the next matching resource object. Repeat from step \ref{check_uri} using the new object.
\end{enumerate}

\item \label{find_method} Identify the method that will handle the request.

\begin{enumerate}
\item Find the set of resource methods that meet the following criteria:
\begin{itemize}
\item If \code{uri} is not empty or equal to \lq/\rq, the method must be annotated with  a URI template that, when transformed into a regular expression using the process described in section \ref{template_to_regex}, matches \code{uri}.
\item The request method is supported. If no methods support the request method an implementation MUST generate a method not allowed response (HTTP 405 status). Note the additional support for \code{HEAD} and \code{OPTIONS} described in section \ref{head_and_options}.
\item The media type of the request entity body (if any) is a supported input data format (see section \ref{declaring_method_capabilities}). If no methods support the media type of the request entity body an implementation MUST generate an unsupported media type response (HTTP 415 status).
\item At least one of the acceptable response entity body media types is a supported output data format (see section \ref{declaring_method_capabilities}). If no methods support one of the acceptable response entity body media types an implementation MUST generate a not acceptable response (HTTP 406 status).
\end{itemize} 
\item Sort the matching set of resource methods using the media type of input data as the primary key and the media type of output data as the secondary key.

Sorting of media types follows the general rule: x/y $<$ x/* $<$ */*, i.e. a method that explicitly lists one of the requested media types is sorted before a method that lists */*. Quality parameter values are also used such that x/y;q=1.0 $<$ x/y;q=0.7.

\item \label{dispatch_method} If the set of matching resource methods is non-empty then the request is dispatched to the first Java method in the set; otherwise no matching resource method can be found and the algorithm terminates.
\end{enumerate}

\end{enumerate}

\subsection{Converting URI Templates to Regular Expressions}
\label{template_to_regex}

A URI template is converted into a regular expression by:
\begin{enumerate}
\item If required, i.e. \UriTemplate\code{.encode=true}, URI encoding any invalid characters in the template, ignoring URI template variable specifications.
\item Escaping any regular expression characters in the URI template, again ignoring URI template variable specifications.
\item Substituting \lq(.*?)\rq\ for each occurrence of a URI template variable (\{$\backslash($[w-$\backslash.$\_~]+?$\backslash)$\}) within the URI template.
\end{enumerate}

Note that the above renders the name of template variables irrelevant for template matching purposes. However, implementations will need to retain template variable names in order to facilitate the extraction of template variable values via \UriParam\ or \UriInfo\code{.getURIParameters}.