\chapter{Environment}
\label{environment}

The container-managed resources available to a \jaxrs\ resource class or provider depend on the environment in which it is deployed. Section \ref{contexttypes} describes the types of context available regardless of container. The following sections describe the additional container-managed resources available to a \jaxrs\ resource class deployed in a variety of environments.

\section{Servlet Container}

The \Context\ annotation can be used to indicate a dependency on a Servlet-defined resource. A Servlet-based implementation MUST support injection of the following Servlet-defined types: \code{Servlet\-Config}, \code{Servlet\-Context}, \code{Http\-Servlet\-Request} and \code{Http\-Servlet\-Response}.

An injected \code{Http\-Servlet\-Request} allows a resource method to stream the contents of a request entity. If the resource method has a parameter whose value is derived from the request entity then the stream will have already been consumed and an attempt to access it MAY result in an exception.

An injected \code{Http\-Servlet\-Response} allows a resource method to commit the HTTP response prior to returning. An implementation MUST check the committed status and only process the return value if the response is not yet committed.

\section{Java EE Container}

\begin{ednote*}
TBD. We anticipate offering the same resource injection capabilities as are provided for a Servlet instance running in a Java EE Web container. In particular we anticipate supporting dependency injection using the following annotations: @Resource, @Resources, @EJB, @EJBs, @WebServiceRef, @WebServiceRefs, @PersistenceContext, @PersistenceContexts, @PersistenceUnit and @PersistenceUnits. We also anticipate supporting the following JSR 250 lifecycle management and security annotations:  @PostConstruct, @PreDestroy, @RunAs, @RolesAllowed, @PermitAll, @DenyAll and @DeclareRoles.
\end{ednote*}

\section{Other}

Other container technologies MAY specify their own set of injectable resources but MUST, at a minimum, support access to the types of context listed in section \ref{contexttypes}.