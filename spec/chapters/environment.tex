\chapter{Environment}

The container-managed resources available to a \jaxrs\ resource class depend on the environment in which the \jaxrs\ resource class is deployed. As described in chapter \ref{context}, all resource classes can access the \UriInfo, \HttpHeaders\ and \PreconditionEvaluator\ contexts regardless of container. The following sections describe the additional container-managed resources available to a \jaxrs\ resource class deployed in a variety of environments.

\section{Servlet Container}

The \Resource\ annotation can be used to indicate a dependency on a Servlet-defined resource. An implementation MUST support injection of the following types: \code{Servlet\-Config}, \code{Servlet\-Context}, \code{Http\-Servlet\-Request} and \code{Http\-Servlet\-Response}.

\section{Java EE Container}

\begin{ednote*}
TBD. We anticipate offering the same resource injection capabilities as are provided for a Servlet instance running in a Java EE Web container. In particular we anticipate supporting dependency injection using the following annotations: @Resource, @Resources, @EJB, @EJBs, @WebServiceRef, @WebServiceRefs, @PersistenceContext, @PersistenceContexts, @PersistenceUnit and @PersistenceUnits. We also anticipate supporting the following JSR 250 lifecycle management and security annotations:  @PostConstruct, @PreDestroy, @RunAs, @RolesAllowed, @PermitAll, @DenyAll and @DeclareRoles.
\end{ednote*}

\section{Other}

Other container technologies MAY specify their own set of injectable resources but MUST, at a minimum, support access to the \UriInfo, \HttpHeaders\ and \PreconditionEvaluator\ as described in chapter \ref{context}.